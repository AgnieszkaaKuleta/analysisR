% Options for packages loaded elsewhere
\PassOptionsToPackage{unicode}{hyperref}
\PassOptionsToPackage{hyphens}{url}
\PassOptionsToPackage{dvipsnames,svgnames,x11names}{xcolor}
%
\documentclass[
  letterpaper,
  DIV=11,
  numbers=noendperiod]{scrreprt}

\usepackage{amsmath,amssymb}
\usepackage{iftex}
\ifPDFTeX
  \usepackage[T1]{fontenc}
  \usepackage[utf8]{inputenc}
  \usepackage{textcomp} % provide euro and other symbols
\else % if luatex or xetex
  \usepackage{unicode-math}
  \defaultfontfeatures{Scale=MatchLowercase}
  \defaultfontfeatures[\rmfamily]{Ligatures=TeX,Scale=1}
\fi
\usepackage{lmodern}
\ifPDFTeX\else  
    % xetex/luatex font selection
\fi
% Use upquote if available, for straight quotes in verbatim environments
\IfFileExists{upquote.sty}{\usepackage{upquote}}{}
\IfFileExists{microtype.sty}{% use microtype if available
  \usepackage[]{microtype}
  \UseMicrotypeSet[protrusion]{basicmath} % disable protrusion for tt fonts
}{}
\makeatletter
\@ifundefined{KOMAClassName}{% if non-KOMA class
  \IfFileExists{parskip.sty}{%
    \usepackage{parskip}
  }{% else
    \setlength{\parindent}{0pt}
    \setlength{\parskip}{6pt plus 2pt minus 1pt}}
}{% if KOMA class
  \KOMAoptions{parskip=half}}
\makeatother
\usepackage{xcolor}
\setlength{\emergencystretch}{3em} % prevent overfull lines
\setcounter{secnumdepth}{5}
% Make \paragraph and \subparagraph free-standing
\ifx\paragraph\undefined\else
  \let\oldparagraph\paragraph
  \renewcommand{\paragraph}[1]{\oldparagraph{#1}\mbox{}}
\fi
\ifx\subparagraph\undefined\else
  \let\oldsubparagraph\subparagraph
  \renewcommand{\subparagraph}[1]{\oldsubparagraph{#1}\mbox{}}
\fi

\usepackage{color}
\usepackage{fancyvrb}
\newcommand{\VerbBar}{|}
\newcommand{\VERB}{\Verb[commandchars=\\\{\}]}
\DefineVerbatimEnvironment{Highlighting}{Verbatim}{commandchars=\\\{\}}
% Add ',fontsize=\small' for more characters per line
\usepackage{framed}
\definecolor{shadecolor}{RGB}{241,243,245}
\newenvironment{Shaded}{\begin{snugshade}}{\end{snugshade}}
\newcommand{\AlertTok}[1]{\textcolor[rgb]{0.68,0.00,0.00}{#1}}
\newcommand{\AnnotationTok}[1]{\textcolor[rgb]{0.37,0.37,0.37}{#1}}
\newcommand{\AttributeTok}[1]{\textcolor[rgb]{0.40,0.45,0.13}{#1}}
\newcommand{\BaseNTok}[1]{\textcolor[rgb]{0.68,0.00,0.00}{#1}}
\newcommand{\BuiltInTok}[1]{\textcolor[rgb]{0.00,0.23,0.31}{#1}}
\newcommand{\CharTok}[1]{\textcolor[rgb]{0.13,0.47,0.30}{#1}}
\newcommand{\CommentTok}[1]{\textcolor[rgb]{0.37,0.37,0.37}{#1}}
\newcommand{\CommentVarTok}[1]{\textcolor[rgb]{0.37,0.37,0.37}{\textit{#1}}}
\newcommand{\ConstantTok}[1]{\textcolor[rgb]{0.56,0.35,0.01}{#1}}
\newcommand{\ControlFlowTok}[1]{\textcolor[rgb]{0.00,0.23,0.31}{#1}}
\newcommand{\DataTypeTok}[1]{\textcolor[rgb]{0.68,0.00,0.00}{#1}}
\newcommand{\DecValTok}[1]{\textcolor[rgb]{0.68,0.00,0.00}{#1}}
\newcommand{\DocumentationTok}[1]{\textcolor[rgb]{0.37,0.37,0.37}{\textit{#1}}}
\newcommand{\ErrorTok}[1]{\textcolor[rgb]{0.68,0.00,0.00}{#1}}
\newcommand{\ExtensionTok}[1]{\textcolor[rgb]{0.00,0.23,0.31}{#1}}
\newcommand{\FloatTok}[1]{\textcolor[rgb]{0.68,0.00,0.00}{#1}}
\newcommand{\FunctionTok}[1]{\textcolor[rgb]{0.28,0.35,0.67}{#1}}
\newcommand{\ImportTok}[1]{\textcolor[rgb]{0.00,0.46,0.62}{#1}}
\newcommand{\InformationTok}[1]{\textcolor[rgb]{0.37,0.37,0.37}{#1}}
\newcommand{\KeywordTok}[1]{\textcolor[rgb]{0.00,0.23,0.31}{#1}}
\newcommand{\NormalTok}[1]{\textcolor[rgb]{0.00,0.23,0.31}{#1}}
\newcommand{\OperatorTok}[1]{\textcolor[rgb]{0.37,0.37,0.37}{#1}}
\newcommand{\OtherTok}[1]{\textcolor[rgb]{0.00,0.23,0.31}{#1}}
\newcommand{\PreprocessorTok}[1]{\textcolor[rgb]{0.68,0.00,0.00}{#1}}
\newcommand{\RegionMarkerTok}[1]{\textcolor[rgb]{0.00,0.23,0.31}{#1}}
\newcommand{\SpecialCharTok}[1]{\textcolor[rgb]{0.37,0.37,0.37}{#1}}
\newcommand{\SpecialStringTok}[1]{\textcolor[rgb]{0.13,0.47,0.30}{#1}}
\newcommand{\StringTok}[1]{\textcolor[rgb]{0.13,0.47,0.30}{#1}}
\newcommand{\VariableTok}[1]{\textcolor[rgb]{0.07,0.07,0.07}{#1}}
\newcommand{\VerbatimStringTok}[1]{\textcolor[rgb]{0.13,0.47,0.30}{#1}}
\newcommand{\WarningTok}[1]{\textcolor[rgb]{0.37,0.37,0.37}{\textit{#1}}}

\providecommand{\tightlist}{%
  \setlength{\itemsep}{0pt}\setlength{\parskip}{0pt}}\usepackage{longtable,booktabs,array}
\usepackage{calc} % for calculating minipage widths
% Correct order of tables after \paragraph or \subparagraph
\usepackage{etoolbox}
\makeatletter
\patchcmd\longtable{\par}{\if@noskipsec\mbox{}\fi\par}{}{}
\makeatother
% Allow footnotes in longtable head/foot
\IfFileExists{footnotehyper.sty}{\usepackage{footnotehyper}}{\usepackage{footnote}}
\makesavenoteenv{longtable}
\usepackage{graphicx}
\makeatletter
\def\maxwidth{\ifdim\Gin@nat@width>\linewidth\linewidth\else\Gin@nat@width\fi}
\def\maxheight{\ifdim\Gin@nat@height>\textheight\textheight\else\Gin@nat@height\fi}
\makeatother
% Scale images if necessary, so that they will not overflow the page
% margins by default, and it is still possible to overwrite the defaults
% using explicit options in \includegraphics[width, height, ...]{}
\setkeys{Gin}{width=\maxwidth,height=\maxheight,keepaspectratio}
% Set default figure placement to htbp
\makeatletter
\def\fps@figure{htbp}
\makeatother
\newlength{\cslhangindent}
\setlength{\cslhangindent}{1.5em}
\newlength{\csllabelwidth}
\setlength{\csllabelwidth}{3em}
\newlength{\cslentryspacingunit} % times entry-spacing
\setlength{\cslentryspacingunit}{\parskip}
\newenvironment{CSLReferences}[2] % #1 hanging-ident, #2 entry spacing
 {% don't indent paragraphs
  \setlength{\parindent}{0pt}
  % turn on hanging indent if param 1 is 1
  \ifodd #1
  \let\oldpar\par
  \def\par{\hangindent=\cslhangindent\oldpar}
  \fi
  % set entry spacing
  \setlength{\parskip}{#2\cslentryspacingunit}
 }%
 {}
\usepackage{calc}
\newcommand{\CSLBlock}[1]{#1\hfill\break}
\newcommand{\CSLLeftMargin}[1]{\parbox[t]{\csllabelwidth}{#1}}
\newcommand{\CSLRightInline}[1]{\parbox[t]{\linewidth - \csllabelwidth}{#1}\break}
\newcommand{\CSLIndent}[1]{\hspace{\cslhangindent}#1}

\KOMAoption{captions}{tableheading}
\makeatletter
\makeatother
\makeatletter
\@ifpackageloaded{bookmark}{}{\usepackage{bookmark}}
\makeatother
\makeatletter
\@ifpackageloaded{caption}{}{\usepackage{caption}}
\AtBeginDocument{%
\ifdefined\contentsname
  \renewcommand*\contentsname{Table of contents}
\else
  \newcommand\contentsname{Table of contents}
\fi
\ifdefined\listfigurename
  \renewcommand*\listfigurename{List of Figures}
\else
  \newcommand\listfigurename{List of Figures}
\fi
\ifdefined\listtablename
  \renewcommand*\listtablename{List of Tables}
\else
  \newcommand\listtablename{List of Tables}
\fi
\ifdefined\figurename
  \renewcommand*\figurename{Figure}
\else
  \newcommand\figurename{Figure}
\fi
\ifdefined\tablename
  \renewcommand*\tablename{Table}
\else
  \newcommand\tablename{Table}
\fi
}
\@ifpackageloaded{float}{}{\usepackage{float}}
\floatstyle{ruled}
\@ifundefined{c@chapter}{\newfloat{codelisting}{h}{lop}}{\newfloat{codelisting}{h}{lop}[chapter]}
\floatname{codelisting}{Listing}
\newcommand*\listoflistings{\listof{codelisting}{List of Listings}}
\makeatother
\makeatletter
\@ifpackageloaded{caption}{}{\usepackage{caption}}
\@ifpackageloaded{subcaption}{}{\usepackage{subcaption}}
\makeatother
\makeatletter
\@ifpackageloaded{tcolorbox}{}{\usepackage[skins,breakable]{tcolorbox}}
\makeatother
\makeatletter
\@ifundefined{shadecolor}{\definecolor{shadecolor}{rgb}{.97, .97, .97}}
\makeatother
\makeatletter
\makeatother
\makeatletter
\makeatother
\ifLuaTeX
  \usepackage{selnolig}  % disable illegal ligatures
\fi
\IfFileExists{bookmark.sty}{\usepackage{bookmark}}{\usepackage{hyperref}}
\IfFileExists{xurl.sty}{\usepackage{xurl}}{} % add URL line breaks if available
\urlstyle{same} % disable monospaced font for URLs
\hypersetup{
  pdftitle={Detección de cuentas fake en Instragram},
  pdfauthor={Agnieszka Kuleta},
  colorlinks=true,
  linkcolor={blue},
  filecolor={Maroon},
  citecolor={Blue},
  urlcolor={Blue},
  pdfcreator={LaTeX via pandoc}}

\title{Detección de cuentas fake en Instragram}
\usepackage{etoolbox}
\makeatletter
\providecommand{\subtitle}[1]{% add subtitle to \maketitle
  \apptocmd{\@title}{\par {\large #1 \par}}{}{}
}
\makeatother
\subtitle{Laboratorio Cientifico projecto}
\author{Agnieszka Kuleta}
\date{2024-08-05}

\begin{document}
\maketitle
\ifdefined\Shaded\renewenvironment{Shaded}{\begin{tcolorbox}[frame hidden, sharp corners, interior hidden, borderline west={3pt}{0pt}{shadecolor}, breakable, boxrule=0pt, enhanced]}{\end{tcolorbox}}\fi

\renewcommand*\contentsname{Table of contents}
{
\hypersetup{linkcolor=}
\setcounter{tocdepth}{2}
\tableofcontents
}
\bookmarksetup{startatroot}

\hypertarget{preface}{%
\chapter*{Preface}\label{preface}}
\addcontentsline{toc}{chapter}{Preface}

\markboth{Preface}{Preface}

En un proyecto determinado utilizaremos un conjunto de datos ``Cuentas
de Instagram Falsas/Spammer/Genuinas'' disponible en Kaggle. Durante el
análisis, utilizaremos las bibliotecas y métodos aprendidos en la
Universidad de Málaga en las clases de laboratorio científico y,
basándonos en ellos, analizaremos el conjunto de datos en cuestión.

\bookmarksetup{startatroot}

\hypertarget{introduction}{%
\chapter{Introduction}\label{introduction}}

\hypertarget{conjunto-de-datos}{%
\section{Conjunto de datos}\label{conjunto-de-datos}}

En un proyecto determinado utilizaremos un conjunto de datos ``Cuentas
de Instagram Falsas/Spammer/Genuinas'' disponible en Kaggle. El contiene
información sobre cuentas de Instagram clasificadas como:

\begin{itemize}
\tightlist
\item
  Cuentas Falsas (Fake Accounts): Incluye cuentas identificadas como no
  auténticas o falsas. Estas cuentas pueden ser operadas por bots o
  personas que intentan aumentar artificialmente su presencia en
  Instagram mediante diversas prácticas deshonestas.
\item
  Cuentas Spammer (Spammer Accounts): Este conjunto de datos contiene
  cuentas identificadas como spammer. Esto significa que son
  responsables de publicar regularmente contenido no deseado o no útil
  en Instagram, a menudo con el propósito de promocionar productos,
  servicios o sitios web.
\item
  Cuentas Genuinas (Genuine Accounts): Estas cuentas se consideran
  auténticas y reales, lo que significa que pertenecen a usuarios reales
  de Instagram que utilizan la plataforma de acuerdo con las reglas y
  políticas del servicio.
\end{itemize}

El conjunto de datos se divide en secciones de prueba y de
entrenamiento. Utilizaremos esta división en una parte posterior del
proyecto. En las primeras partes del proyecto, nos centraremos en el
seccion de prueba. En la parte de Machine Learning vamos a usar ambos
conjuntos de datos.

Ambos datasets son el tipo de data frame, entonces no tenemos que
cambiar nada. Dataset de test tiene 120 observaciones y 12 variables.
Por otro lado dataset de train contiene 576 observaciones y 12 los
mismos varaibales.

\hypertarget{libraries-que-usamos}{%
\section{Libraries que usamos}\label{libraries-que-usamos}}

Utilizaremos las bibliotecas que se indican a continuación. Estas
bibliotecas nos permitirán visualizar e interpretar los datos (ggplot2,
gganimate, dplyr), para hacer reglas(arules) y para ml caret.

\hypertarget{objetivos-del-proyecto}{%
\section{Objetivos del proyecto}\label{objetivos-del-proyecto}}

\begin{enumerate}
\def\labelenumi{\arabic{enumi}.}
\tightlist
\item
  Detección de cuentas falsas y de spam: desarrollar modelos de
  aprendizaje automático que detecten automáticamente cuentas falsas y
  de spam en la plataforma Instagram. 2.Prevención del spam y la
  manipulación: identificar reglas de spam y manipulación.
\end{enumerate}

\bookmarksetup{startatroot}

\hypertarget{anuxe1lisis-exploratorio-de-datos}{%
\chapter{Análisis exploratorio de
datos}\label{anuxe1lisis-exploratorio-de-datos}}

\hypertarget{cargand-data-set-y-libraries}{%
\section{Cargand data set y
libraries}\label{cargand-data-set-y-libraries}}

El primer paso para empezar con este proyecto es cargar los datos y
comprender cómo está estructurada la información dentro del conjunto de
datos.

\begin{Shaded}
\begin{Highlighting}[]
\FunctionTok{library}\NormalTok{(readr)}
\FunctionTok{library}\NormalTok{(ggplot2)}
\end{Highlighting}
\end{Shaded}

\begin{verbatim}
Warning: package 'ggplot2' was built under R version 4.3.3
\end{verbatim}

\begin{Shaded}
\begin{Highlighting}[]
\FunctionTok{library}\NormalTok{(gganimate)}
\end{Highlighting}
\end{Shaded}

\begin{verbatim}
Warning: package 'gganimate' was built under R version 4.3.3
\end{verbatim}

\begin{Shaded}
\begin{Highlighting}[]
\FunctionTok{library}\NormalTok{(dplyr)}
\end{Highlighting}
\end{Shaded}

\begin{verbatim}

Attaching package: 'dplyr'
\end{verbatim}

\begin{verbatim}
The following objects are masked from 'package:stats':

    filter, lag
\end{verbatim}

\begin{verbatim}
The following objects are masked from 'package:base':

    intersect, setdiff, setequal, union
\end{verbatim}

\begin{Shaded}
\begin{Highlighting}[]
\NormalTok{datatest}\OtherTok{\textless{}{-}}\FunctionTok{read.csv}\NormalTok{(}\StringTok{"test.csv"}\NormalTok{,}\AttributeTok{sep =} \StringTok{","}\NormalTok{)}
\NormalTok{datatrain}\OtherTok{\textless{}{-}}\FunctionTok{read.csv}\NormalTok{(}\StringTok{"train.csv"}\NormalTok{, }\AttributeTok{sep =} \StringTok{","}\NormalTok{)}
\end{Highlighting}
\end{Shaded}

\hypertarget{observaciones-del-conjunto-de-datos}{%
\section{Observaciones del conjunto de
datos}\label{observaciones-del-conjunto-de-datos}}

A continuación, aplicamos todos los cambios necesarios al conjunto de
datos antes de trabajar en el análisis exploratorio de datos y el
modelado.

\begin{Shaded}
\begin{Highlighting}[]
\FunctionTok{head}\NormalTok{(datatrain)}
\end{Highlighting}
\end{Shaded}

\begin{verbatim}
  profile.pic nums.length.username fullname.words nums.length.fullname
1           1                 0.27              0                    0
2           1                 0.00              2                    0
3           1                 0.10              2                    0
4           1                 0.00              1                    0
5           1                 0.00              2                    0
6           1                 0.00              4                    0
  name..username description.length external.URL private X.posts X.followers
1              0                 53            0       0      32        1000
2              0                 44            0       0     286        2740
3              0                  0            0       1      13         159
4              0                 82            0       0     679         414
5              0                  0            0       1       6         151
6              0                 81            1       0     344      669987
  X.follows fake
1       955    0
2       533    0
3        98    0
4       651    0
5       126    0
6       150    0
\end{verbatim}

\begin{Shaded}
\begin{Highlighting}[]
\FunctionTok{tail}\NormalTok{(datatrain)}
\end{Highlighting}
\end{Shaded}

\begin{verbatim}
    profile.pic nums.length.username fullname.words nums.length.fullname
571           1                 0.20              1                 0.00
572           1                 0.55              1                 0.44
573           1                 0.38              1                 0.33
574           1                 0.57              2                 0.00
575           1                 0.57              1                 0.00
576           1                 0.27              1                 0.00
    name..username description.length external.URL private X.posts X.followers
571              0                 28            0       0       0          15
572              0                  0            0       0      33         166
573              0                 21            0       0      44          66
574              0                  0            0       0       4          96
575              0                 11            0       0       0          57
576              0                  0            0       0       2         150
    X.follows fake
571        64    1
572       596    1
573        75    1
574       339    1
575        73    1
576       487    1
\end{verbatim}

\hypertarget{na-valores}{%
\subsection{NA valores}\label{na-valores}}

\begin{Shaded}
\begin{Highlighting}[]
\FunctionTok{print}\NormalTok{(}\StringTok{"Data train:"}\NormalTok{)}
\end{Highlighting}
\end{Shaded}

\begin{verbatim}
[1] "Data train:"
\end{verbatim}

\begin{Shaded}
\begin{Highlighting}[]
\FunctionTok{sum}\NormalTok{(}\FunctionTok{is.na.data.frame}\NormalTok{(datatrain))}
\end{Highlighting}
\end{Shaded}

\begin{verbatim}
[1] 0
\end{verbatim}

\begin{Shaded}
\begin{Highlighting}[]
\FunctionTok{print}\NormalTok{(}\StringTok{"Data test:"}\NormalTok{)}
\end{Highlighting}
\end{Shaded}

\begin{verbatim}
[1] "Data test:"
\end{verbatim}

\begin{Shaded}
\begin{Highlighting}[]
\FunctionTok{sum}\NormalTok{(}\FunctionTok{is.na.data.frame}\NormalTok{(datatest))}
\end{Highlighting}
\end{Shaded}

\begin{verbatim}
[1] 0
\end{verbatim}

En primer lugar, podemos ver que no hay valores perdidos que tratar.
Todas las características de ambos marcos de datos son numéricas. Ambos
marcos de datos son pequeños, y el mayor de ellos sólo tiene 576
muestras de datos.

\begin{Shaded}
\begin{Highlighting}[]
\FunctionTok{summary}\NormalTok{(datatrain)}
\end{Highlighting}
\end{Shaded}

\begin{verbatim}
  profile.pic     nums.length.username fullname.words  nums.length.fullname
 Min.   :0.0000   Min.   :0.0000       Min.   : 0.00   Min.   :0.00000     
 1st Qu.:0.0000   1st Qu.:0.0000       1st Qu.: 1.00   1st Qu.:0.00000     
 Median :1.0000   Median :0.0000       Median : 1.00   Median :0.00000     
 Mean   :0.7014   Mean   :0.1638       Mean   : 1.46   Mean   :0.03609     
 3rd Qu.:1.0000   3rd Qu.:0.3100       3rd Qu.: 2.00   3rd Qu.:0.00000     
 Max.   :1.0000   Max.   :0.9200       Max.   :12.00   Max.   :1.00000     
 name..username    description.length  external.URL       private      
 Min.   :0.00000   Min.   :  0.00     Min.   :0.0000   Min.   :0.0000  
 1st Qu.:0.00000   1st Qu.:  0.00     1st Qu.:0.0000   1st Qu.:0.0000  
 Median :0.00000   Median :  0.00     Median :0.0000   Median :0.0000  
 Mean   :0.03472   Mean   : 22.62     Mean   :0.1163   Mean   :0.3819  
 3rd Qu.:0.00000   3rd Qu.: 34.00     3rd Qu.:0.0000   3rd Qu.:1.0000  
 Max.   :1.00000   Max.   :150.00     Max.   :1.0000   Max.   :1.0000  
    X.posts        X.followers         X.follows           fake    
 Min.   :   0.0   Min.   :       0   Min.   :   0.0   Min.   :0.0  
 1st Qu.:   0.0   1st Qu.:      39   1st Qu.:  57.5   1st Qu.:0.0  
 Median :   9.0   Median :     150   Median : 229.5   Median :0.5  
 Mean   : 107.5   Mean   :   85307   Mean   : 508.4   Mean   :0.5  
 3rd Qu.:  81.5   3rd Qu.:     716   3rd Qu.: 589.5   3rd Qu.:1.0  
 Max.   :7389.0   Max.   :15338538   Max.   :7500.0   Max.   :1.0  
\end{verbatim}

Se puede observar que nuestros marcos de datos están organizados en lo
que parecen ser características continuas y binarias. Estas son las
características que tenemos a mano:

\begin{itemize}
\item
  foto de perfil: variable binaria que indica si una cuenta tiene una
  foto de perfil(1) o no(0);
\item
  nums/length username: variable continua que indica la proporción de
  caracteres numéricos por la longitud total del nombre de usuario de
  una cuenta;
\item
  palabras de nombre completo: variable continua que cuenta el total de
  palabras del nombre de la persona titular de la cuenta;
\item
  nums/lenght fullname: variable continua que indica la proporción de
  caracteres numéricos por la longitud total del nombre completo de la
  persona;
\item
  name == username: variable binaria que indica si el nombre de la
  persona coincide con el nombre de usuario (1=si, 0=no);
\item
  longitud de la descripción: variable continua, la longitud de la
  descripción del perfil.
\item
  URL externa: variable binaria que indica si un perfil tiene un enlace
  a un sitio web externo(1) en su bio(0);
\item
  privado: variable binaria que indica si el perfil está cerrado(1) para
  los no seguidores o no(0);
\item
  \#posts: variable continua que contiene el número de publicaciones de
  ese perfil;
\item
  \#seguidores: variable continua que contiene el número total de
  seguidores de cada cuenta;
\item
  \#seguidos:variable continua que contiene el número total de personas
  a las que sigue la cuenta
\item
  fake: La variable objetivo.Si una cuenta es falsa(1) o no(0).
\end{itemize}

\hypertarget{duplicados}{%
\subsection{Duplicados}\label{duplicados}}

Las filas duplicadas pueden introducir sesgos en el modelo y llevar a un
sobreajuste. Así que vamos a eliminarlas.

\begin{Shaded}
\begin{Highlighting}[]
\NormalTok{datatest }\OtherTok{\textless{}{-}}\NormalTok{ datatest }\SpecialCharTok{\%\textgreater{}\%}
  \FunctionTok{group\_by\_all}\NormalTok{() }\SpecialCharTok{\%\textgreater{}\%}
  \FunctionTok{filter}\NormalTok{(}\FunctionTok{row\_number}\NormalTok{() }\SpecialCharTok{==} \DecValTok{1}\NormalTok{)}

\NormalTok{datatrain }\OtherTok{\textless{}{-}}\NormalTok{ datatrain }\SpecialCharTok{\%\textgreater{}\%}
  \FunctionTok{group\_by\_all}\NormalTok{() }\SpecialCharTok{\%\textgreater{}\%}
  \FunctionTok{filter}\NormalTok{(}\FunctionTok{row\_number}\NormalTok{() }\SpecialCharTok{==} \DecValTok{1}\NormalTok{)}
\FunctionTok{print}\NormalTok{(}\FunctionTok{paste}\NormalTok{(}\StringTok{"Data train:"}\NormalTok{, }\FunctionTok{dim}\NormalTok{(datatrain)))}
\end{Highlighting}
\end{Shaded}

\begin{verbatim}
[1] "Data train: 574" "Data train: 12" 
\end{verbatim}

\begin{Shaded}
\begin{Highlighting}[]
\FunctionTok{print}\NormalTok{(}\FunctionTok{paste}\NormalTok{(}\StringTok{"Data test:"}\NormalTok{, }\FunctionTok{dim}\NormalTok{(datatest)))}
\end{Highlighting}
\end{Shaded}

\begin{verbatim}
[1] "Data test: 118" "Data test: 12" 
\end{verbatim}

\bookmarksetup{startatroot}

\hypertarget{anuxe1lisis-exploratorio-de-datos-y-visualisaciones}{%
\chapter{Análisis exploratorio de datos y
visualisaciones}\label{anuxe1lisis-exploratorio-de-datos-y-visualisaciones}}

\hypertarget{outliers}{%
\subsection{outliers}\label{outliers}}

\bookmarksetup{startatroot}

\hypertarget{reglas}{%
\chapter{reglas}\label{reglas}}

\bookmarksetup{startatroot}

\hypertarget{fcar}{%
\chapter{fcar}\label{fcar}}

\bookmarksetup{startatroot}

\hypertarget{machine-learning}{%
\chapter{Machine Learning}\label{machine-learning}}

\bookmarksetup{startatroot}

\hypertarget{linear-regression}{%
\chapter{Linear regression}\label{linear-regression}}

\bookmarksetup{startatroot}

\hypertarget{time-series}{%
\chapter{Time series}\label{time-series}}

\hypertarget{decision-tree}{%
\section{Decision tree}\label{decision-tree}}

\hypertarget{clustering}{%
\section{Clustering}\label{clustering}}

\bookmarksetup{startatroot}

\hypertarget{summary}{%
\chapter{Summary}\label{summary}}

In summary, this book has no content whatsoever.

\begin{Shaded}
\begin{Highlighting}[]
\DecValTok{1} \SpecialCharTok{+} \DecValTok{1}
\end{Highlighting}
\end{Shaded}

\begin{verbatim}
[1] 2
\end{verbatim}

\bookmarksetup{startatroot}

\hypertarget{references}{%
\chapter*{References}\label{references}}
\addcontentsline{toc}{chapter}{References}

\markboth{References}{References}

\hypertarget{refs}{}
\begin{CSLReferences}{0}{0}
\end{CSLReferences}



\end{document}
